% Copyright 2021 Indigo BioAutomation, Inc.
%
% Permission is hereby granted, free of charge, to any person
% obtaining a copy of this software and associated documentation files
% (the "Software"), to deal in the Software without restriction,
% including without limitation the rights to use, copy, modify, merge,
% publish, distribute, sublicense, and/or sell copies of the Software,
% and to permit persons to whom the Software is furnished to do so,
% subject to the following conditions:
%
% The above copyright notice and this permission notice shall be
% included in all copies or substantial portions of the Software.
%
% THE SOFTWARE IS PROVIDED "AS IS", WITHOUT WARRANTY OF ANY KIND,
% EXPRESS OR IMPLIED, INCLUDING BUT NOT LIMITED TO THE WARRANTIES OF
% MERCHANTABILITY, FITNESS FOR A PARTICULAR PURPOSE AND
% NONINFRINGEMENT. IN NO EVENT SHALL THE AUTHORS OR COPYRIGHT HOLDERS
% BE LIABLE FOR ANY CLAIM, DAMAGES OR OTHER LIABILITY, WHETHER IN AN
% ACTION OF CONTRACT, TORT OR OTHERWISE, ARISING FROM, OUT OF OR IN
% CONNECTION WITH THE SOFTWARE OR THE USE OR OTHER DEALINGS IN THE
% SOFTWARE.

\chapter {Stream Library}\label{chap:stream}

\section {Introduction}

Streams facilitate composition of higher-order functions like map, filter, and fold, while
minimizing allocation of intermediate data structures. The \code{s/>} procedure allows
expressions analogous to those found in \csharp, Ruby, and other popular languages, in
which a stream of values is passed through a pipeline of operators.

A \emph{stream}\index{stream} is a procedure of zero arguments that advances the position
within the stream and returns the next value or the end-of-stream object \emph{eos}.

A \emph{transformer}\index{transformer} is a procedure that takes a single argument and
returns a value.  The argument and return value are often streams. If a transformer
requires its argument to be a stream, it is responsible for validating or coercing it. A
well-behaved transformer can be applied safely to multiple streams or repeatedly to the
same stream. All transformers defined in this library are well-behaved.

A \emph{transformer constructor} is a procedure that returns a transformer. \code{s/map}
is one such constructor. \code{(s/map f)} returns a transformer that maps stream values
with procedure \code{f}.

Two transformers can be composed into a new transformer. The \code{s/>} procedure composes
zero or more transformers and then applies them to an argument.

\codebegin
> (s/> '(1 2 3) (s/map 1+))
(2 3 4)
\codeend

\section {Basics}

\defineentry{s/>}
\begin{procedure}
  \code{(s/> \var{x} \var{t} \ldots)}
\end{procedure}
\returns{} see below

The \code{s/>} procedure applies transformers \var{t} \ldots\ to \var{x} in order. If
\var{x} is a list, vector, or hashtable, it is converted to a stream. The final
transformation result is passed to \code{unstream} before returning.

\code{s/>} is pronounced ``stream pipe.''

\defineentry{stream}
\begin{procedure}
  \code{(stream \var{x} \ldots)}
\end{procedure}
\returns{} a stream

The \code{stream} procedure returns a stream containing values \var{x} \ldots.

\defineentry{list->stream}
\begin{procedure}
  \code{(list->stream \var{list})}
\end{procedure}
\returns{} a stream

The \code{list->stream} procedure returns a stream containing the values in \var{list}, in
order.

\defineentry{vector->stream}
\begin{procedure}
  \code{(vector->stream \var{vector})}
\end{procedure}
\returns{} a stream

The \code{vector->stream} procedure returns a stream containing the values in
\var{vector}, in order.

\defineentry{hashtable->stream}
\begin{procedure}
  \code{(hashtable->stream \var{hashtable})}
\end{procedure}
\returns{} a stream

The \code{hashtable->stream} procedure returns a stream containing the cells of
\var{hashtable} in no particular order.

\defineentry{stream->list}
\begin{procedure}
  \code{(stream->list \var{stream})}
\end{procedure}
\returns{} a list

The \code{stream->list} procedure returns a list containing the values in a finite
\var{stream}, in order.

\defineentry{stream-unfold}
\begin{procedure}
  \code{(stream-unfold \var{procedure} \var{state})}
\end{procedure}
\returns{} a stream

The \code{stream-unfold} procedure returns a stream formed by the initial \var{state} and
repeated calls to \var{procedure}. \var{procedure} is called with the current state and
should return zero, one, or two values. Zero values indicate the end of the stream. One
value indicates the last value in the stream. Two values indicate the next value and the
next state.

\codebegin
> (define (numbers-from n)
    (stream-unfold (lambda (n) (values n (+ n 1))) n))
> (s/> (numbers-from 1) (s/take 3))
(1 2 3)
\codeend

\codebegin
> (define (range a b step)
    (stream-unfold
     (lambda (n)
       (cond
        [(< n b) (values n (+ n step))]
        [(= n b) (values n)]
        [else (values)]))
     a))
> (stream->list (range 10 30 5))
(10 15 20 25 30)
\codeend

\defineentry{stream-repeat}
\begin{procedure}
  \code{(stream-repeat \var{n} \var{procedure})}
\end{procedure}
\returns{} a stream

The \code{stream-repeat} procedure returns a stream of \var{n} values. The
\var{i}$^\textrm{th}$ value is \code{(procedure \var{i})}. \var{procedure} is invoked in
order with \var{i} starting at 0.

\codebegin
> (stream->list (stream-repeat 5 values))
(0 1 2 3 4)
\codeend

\defineentry{stream-lift}
\begin{procedure}
  \code{(stream-lift \var{p})}
\end{procedure}
\returns{} a procedure

The \code{stream-lift} procedure returns a procedure that converts its argument with
\code{unstream} and then passes it to \var{p}. It can be used to include a
non-stream-aware procedure in a stream pipeline.

\defineentry{unstream}
\begin{procedure}
  \code{(unstream \var{x})}
\end{procedure}
\returns{} see below

If \var{x} is a stream, it returns \code{(stream->list \var{x})}. If \var{x} is a stream
box (see \code{s/stream}), it is unboxed. Otherwise, it returns \var{x}.

\section {Transformers}

\defineentry{s/all?}
\begin{procedure}
  \code{(s/all? \var{procedure})}
\end{procedure}
\returns{} a transformer

The \code{s/all?} procedure returns a transformer that returns true if and only if
\var{procedure} returns true for all values in the finite stream.

\defineentry{s/any}
\begin{variable}
  \code{s/any}
\end{variable}
\antipar

The \code{s/any} variable is a transformer that returns false if the stream is empty, and
the stream itself if non-empty.

\defineentry{s/any?}
\begin{procedure}
  \code{(s/any? \opt{\var{procedure}})}
\end{procedure}
\returns{} a transformer

The \code{s/any?} procedure returns a transformer that returns true if and only if the
stream has at least one value. If \var{procedure} is provided, the stream is first
filtered with \var{procedure}.

\defineentry{s/by-key}
\begin{procedure}
  \code{(s/by-key \var{procedure} \opt{\var{hash} \var{equiv?}})}
\end{procedure}
\returns{} a transformer

The \code{s/by-key} procedure returns a transformer that indexes the values of a finite
stream by key function \var{procedure} into a hashtable created with \code{(make-hashtable
  \var{hash} \var{equiv?})}. If \var{hash} and \var{equiv?} are not provided, a hashtable
appropriate for the first key is created (see Table~\ref{tab:stream-ht}). If two values
$v_1$ and $v_2$ have the same key \var{k}, error \code{\#(duplicate-key \var{k} $v_1$
  $v_2$)} is raised.

\begin{table}[H]
  \center
  \begin{tabular}{ll}
    Key Type & Constructor \\ \hline
    fixnum, record, or boolean & \code{(make-eq-hashtable)} \\
    non-fixnum number or char & \code{(make-eqv-hashtable)} \\
    string & \code{(make-hashtable string-hash string=?)} \\
    symbol & \code{(make-hashtable symbol-hash eq?)} \\
    else & \code{(make-hashtable equal-hash equal?)}\\
    \hline
  \end{tabular}
  \caption{Hashtable Type Inference}\label{tab:stream-ht}
\end{table}

\defineentry{s/cells}
\begin{variable}
  \code{s/cells}
\end{variable}
\antipar

The \code{s/cells} variable is a transformer that returns a stream of cells
\code{(\var{k}~.~\var{v})} of the given hashtable in no particular order. If the input
value is not a hashtable, it is returned.

\defineentry{s/chunk}
\begin{procedure}
  \code{(s/chunk \var{procedure})}
\end{procedure}
\returns{} a transformer

The \code{s/chunk} procedure returns a transformer that groups stream values into lists
based on the return value of \var{procedure}. Consecutive values \var{v} \ldots\ for which
\code{(procedure v)} returns the same value are grouped into a list. Return values are
compared with \code{equal?}.

\codebegin
> (s/> '(1 3 4) (s/chunk odd?))
((1 3) (4))
> (s/> '("one" "two" "three" "four" "five" "six") (s/chunk string-length))
(("one" "two") ("three") ("four" "five") ("six"))
\codeend

\defineentry{s/chunk2}
\begin{procedure}
  \code{(s/chunk2 \var{procedure})}
\end{procedure}
\returns{} a transformer

The \code{s/chunk2} procedure returns a transformer that groups stream values into lists
based on the return value of \var{procedure}. Consecutive value pairs $v_1$ and $v_2$ for
which \code{(procedure $v_1$ $v_2$)} returns the same value are grouped into a
list. Return values are compared with \code{equal?}.

\codebegin
> (s/> '(2 3 4 3 2 1 2 3 4) (s/chunk2 <=))
((2 3 4) (3 2 1) (2 3 4))
\codeend

\defineentry{s/chunk-every}
\begin{procedure}
  \code{(s/chunk-every $n$)}
\end{procedure}
\returns{} a transformer

The \code{s/chunk-every} procedure returns a transformer that groups stream values into
lists of length $n$. The last list length $L$ is $0 < L \le n$. $n$ must be an exact,
positive integer.

\codebegin
> (s/> (iota 5) (s/chunk-every 2))
((0 1) (2 3) (4))
\codeend

\defineentry{s/concat}
\begin{variable}
  \code{s/concat}
\end{variable}
\antipar

The \code{s/concat} variable is a transformer that concatenates a heterogeneous stream of
streams, lists, or vectors.

\codebegin
> (s/> (list '(1 2) '#(3 4) (stream 5 6)) s/concat)
(1 2 3 4 5 6)
\codeend

\defineentry{s/count}
\begin{procedure}
  \code{(s/count \var{procedure})}
\end{procedure}
\returns{} a transformer

The \code{s/count} procedure returns a transformer that applies \var{procedure} to each
value in a finite stream and sums the results, treating false as 0 and other non-numbers
as 1. For example, if \var{procedure} returns a boolean, \code{s/count} counts the number
of values for which \var{procedure} returns true.

\codebegin
> (s/> '(1 2 3) (s/count odd?))
2
> (s/> '("1" "two" "3") (s/count string->number))
4
\codeend

\defineentry{s/do}
\begin{procedure}
  \code{(s/do \var{procedure})}
\end{procedure}
\returns{} a transformer

The \code{s/do} procedure returns a transformer that invokes \var{procedure} on every
value in the stream, in order, and returns an unspecified value if the stream is finite.

\defineentry{s/drop}
\begin{procedure}
  \code{(s/drop $n$)}
\end{procedure}
\returns{} a transformer

The \code{s/drop} procedure returns a transformer that returns the $n^\textrm{th}$ tail of
a stream. $n$ must be an exact integer. If $n$ is negative, it returns the input stream
unchanged. If $n$ is greater than or equal to the length of the stream, it returns the
empty stream.

\defineentry{s/drop-last}
\begin{procedure}
  \code{(s/drop-last $n$)}
\end{procedure}
\returns{} a transformer

The \code{s/drop-last} procedure returns a transformer that returns a stream containing
all but the last $n$ values in a finite input stream. $n$ must be an exact integer. If $n$
is negative, it returns the input stream unchanged. If $n$ is greater than or equal to the
length of the stream, it returns the empty stream.

\defineentry{s/drop-while}
\begin{procedure}
  \code{(s/drop-while \var{procedure})}
\end{procedure}
\returns{} a transformer

The \code{s/drop-while} procedure returns a transformer that returns the tail of the
stream starting with the first value for which \var{procedure} returns false. If
\var{procedure} returns a true value for all values, it returns the empty stream.

\codebegin
> (s/> '(1 3 4 5) (s/drop-while odd?))
(4 5)
\codeend

\defineentry{s/extrema}
\begin{procedure}
  \code{(s/extrema \opt{\var{lt} \var{gt}})}
\end{procedure}
\returns{} a transformer

The \code{s/extrema} procedure returns a transformer that performs \code{s/min} and
\code{s/max} simultaneously, returning the list \code{(\var{min} \var{max})}, or false if
the stream is empty.

\defineentry{s/extrema-by}
\begin{procedure}
  \code{(s/extrema-by \var{procedure} \opt{\var{lt} \var{gt}})}
\end{procedure}
\returns{} a transformer

The \code{s/extrema-by} procedure returns a transformer that performs \code{s/min-by} and
\code{s/max-by} simultaneously, returning the list \code{(\var{min} \var{max})}, or false
if the stream is empty.

\defineentry{s/filter}
\begin{procedure}
  \code{(s/filter \opt{\var{procedure}})}
\end{procedure}
\returns{} a transformer

The \code{s/filter} procedure returns a transformer that filters stream values by
\var{procedure}. If \var{procedure} is not provided, the identity procedure is used.

\defineentry{s/find}
\begin{procedure}
  \code{(s/find \var{procedure})}
\end{procedure}
\returns{} a transformer

The \code{s/find} procedure returns the transformer composition of \code{(s/filter
  \var{procedure})} and \code{s/first}.

\defineentry{s/first}
\begin{variable}
  \code{s/first}
\end{variable}
\antipar

The \code{s/first} variable is a transformer that returns the first value in the stream, or
false if the stream is empty.

\defineentry{s/fold-left}
\begin{procedure}
  \code{(s/fold-left \var{procedure} \var{acc})}
\end{procedure}
\returns{} a transformer

The \code{s/fold-left} procedure returns a transformer that folds \var{procedure} over a
finite stream with initial accumulator \var{acc}. The semantics are the same as
\code{fold-left}.

\defineentry{s/fold-right}
\begin{procedure}
  \code{(s/fold-right \var{procedure} \var{acc})}
\end{procedure}
\returns{} a transformer

The \code{s/fold-right} procedure returns a transformer that folds \var{procedure} over a
finite stream with initial accumulator \var{acc}. The semantics are the same as
\code{fold-right}.

\defineentry{s/group-by}
\begin{procedure}
  \code{(s/group-by \var{procedure} \opt{\var{hash} \var{equiv?}})}
\end{procedure}
\returns{} a transformer

The \code{s/group-by} procedure returns a transformer that groups the values of a finite
stream by key function \var{procedure} into a hashtable created with \code{(make-hashtable
  \var{hash} \var{equiv?})}. The value for key \var{k} is the list of stream values for
which \var{procedure} returns \var{k}, in order of their appearance in the stream. If
\var{hash} and \var{equiv?} are not provided, a hashtable appropriate for the first key is
created (see Table~\ref{tab:stream-ht}).

\defineentry{s/ht}
\begin{procedure}
  \code{(s/ht \var{fk} \var{fv} \opt{\var{hash} \var{equiv?}})}
\end{procedure}
\returns{} a transformer

The \code{s/ht} procedure returns a transformer that maps each value \var{x} in a finite
stream to a cell with key \code{(fk x)} and value \code{(fv x)} in a hashtable created
with \code{(make-hashtable \var{hash} \var{equiv?})}. If \var{hash} and \var{equiv?} are
not provided, a hashtable appropriate for the first key is created (see
Table~\ref{tab:stream-ht}).  If two values $v_1$ and $v_2$ have the same key \var{k},
error \code{\#(duplicate-key \var{k} $v_1$ $v_2$)} is raised.

\codebegin
> (s/> '(1 2 3)
    (s/ht 1- 1+)
    s/cells
    (s/sort-by car <))
((0 . 2) (1 . 3) (2 . 4))
\codeend

\defineentry{s/ht*}
\begin{procedure}
  \code{(s/ht* \var{fk} \var{fv} \opt{\var{hash} \var{equiv?}})}
\end{procedure}
\returns{} a transformer

The \code{s/ht*} procedure returns a transformer that maps each value in a finite stream
into a hashtable created with \code{(make-hashtable \var{hash} \var{equiv?})}. Stream
value \var{x} is mapped to key \code{(fk x)}. The value for a key is the list of stream
values with that key, in order of their appearance in the stream, mapped with \var{fv}. If
\var{hash} and \var{equiv?} are not provided, a hashtable appropriate for the first key is
created (see Table~\ref{tab:stream-ht}).

\codebegin
> (s/> '("a" "b" "cc")
    (s/ht* string-length string->symbol)
    s/cells
    (s/sort-by car <))
((1 a b) (2 cc))
\codeend

\defineentry{s/ht**}
\begin{procedure}
  \code{(s/ht** \var{fk} \var{fv} \opt{\var{hash} \var{equiv?}})}
\end{procedure}
\returns{} a transformer

The \code{s/ht**} procedure returns a transformer that maps each value in a finite stream
into a hashtable created with \code{(make-hashtable \var{hash} \var{equiv?})}. Stream
value \var{x} is mapped to key \code{(fk x)}. The value for a key is the list of stream
values with that key, in order of their appearance in the stream, map-concatenated with
\var{fv}. If \var{hash} and \var{equiv?} are not provided, a hashtable appropriate for the
first key is created (see Table~\ref{tab:stream-ht}).

\codebegin
> (s/> '(("a" 1 2) ("b" 3 4) ("a" 5 6))
    (s/ht** car cdr)
    s/cells
    (s/sort-by car string<?))
(("a" 1 2 5 6) ("b" 3 4))
\codeend

\defineentry{s/join}
\begin{procedure}
  \code{(s/join \var{sep} \opt{\var{sep-2} \var{sep-last}})}
\end{procedure}
\returns{} a transformer

The \code{s/join} procedure returns a transformer that joins the values of a finite stream
into a string, separated by string \var{sep}. If strings \var{sep-2} and \var{sep-last}
are provided, \var{sep-2} is used to separate a stream of two values, and \var{sep-last}
is used as the last separator for a stream of three or more values.  Values are written to
the string as with \code{display}. It returns the empty string if the stream is empty.

\codebegin
> (s/> '(a b c) (s/join ", "))
"a, b, c"
> (s/> '(a b c) (s/join ", " " and " ", and "))
"a, b, and c"
> (s/> '(a b) (s/join ", " " and " ", and "))
"a and b"
\codeend

\defineentry{s/last}
\begin{variable}
  \code{s/last}
\end{variable}
\antipar

The \code{s/last} variable is a transformer that returns the last value in a finite stream,
or false if the stream is empty.

\defineentry{s/length}
\begin{variable}
  \code{s/length}
\end{variable}
\antipar

The \code{s/length} variable is a transformer that returns the number of values in a
finite stream.

\defineentry{s/map}
\begin{procedure}
  \code{(s/map \var{procedure})}
\end{procedure}
\returns{} a transformer

The \code{s/map} procedure returns a transformer that maps stream values with
\var{procedure}.

\defineentry{s/map-car}
\begin{procedure}
  \code{(s/map-car \var{f})}
\end{procedure}
\returns{} a transformer

The \code{s/map-car} procedure returns a transformer that maps each pair \code{(\var{x}~.
  \var{y})} to \code{($f(x)$~.~\var{y})}.

\codebegin
> (s/> '((a . 1) (b . 2)) (s/map-car symbol->string))
(("a" . 1) ("b" . 2))
\codeend

\defineentry{s/map-cdr}
\begin{procedure}
  \code{(s/map-cdr \var{f})}
\end{procedure}
\returns{} a transformer

The \code{s/map-cdr} procedure returns a transformer that maps each pair
\code{(\var{x}~.~\var{y})} to \code{(\var{x}~.~$f(y)$)}.

\codebegin
> (s/> '((a . 1) (b . 2)) (s/map-cdr 1+))
((a . 2) (b . 3))
\codeend

\defineentry{s/map-concat}
\begin{procedure}
  \code{(s/map-concat \var{procedure})}
\end{procedure}
\returns{} a transformer

The \code{s/map-concat} procedure returns the transformer composition of \code{(s/map
  \var{procedure})} and \code{s/concat}.

\defineentry{s/map-cons}
\begin{procedure}
  \code{(s/map-cons \var{f})}
\end{procedure}
\returns{} a transformer

The \code{s/map-cons} procedure returns a transformer that maps each value \var{x} to the
pair \code{(\var{x}~.~$f(x)$)}.

\codebegin
> (s/> '(1 2) (s/map-cons -))
((1 . -1) (2 . -2))
\codeend

\defineentry{s/map-extrema}
\begin{procedure}
  \code{(s/map-extrema \var{procedure} \opt{\var{lt} \var{gt}})}
\end{procedure}
\returns{} a transformer

The \code{s/map-extrema} procedure returns a transformer that performs \code{s/map-min}
and \code{s/map-max} simultaneously, returning the list \code{(\var{min} \var{max})}, or
false if the stream is empty.

\defineentry{s/map-filter}
\begin{procedure}
  \code{(s/map-filter \var{procedure})}
\end{procedure}
\returns{} a transformer

The \code{s/map-filter} procedure returns the transformer composition of \code{(s/map
  \var{procedure})} and \code{(s/filter)}.

\codebegin
> (s/> '("1" "two" "3") (s/map-filter string->number))
(1 3)
\codeend

\defineentry{s/map-max}
\begin{procedure}
  \code{(s/map-max \var{procedure} \opt{\var{gt}})}
\end{procedure}
\returns{} a transformer

The \code{s/map-max} procedure returns the transformer composition of \code{(s/map
  \var{procedure})} and \code{(s/max \opt{\var{gt}})}.

\defineentry{s/map-min}
\begin{procedure}
  \code{(s/map-min \var{procedure} \opt{\var{lt}})}
\end{procedure}
\returns{} a transformer

The \code{s/map-min} procedure returns the transformer composition of \code{(s/map
  \var{procedure})} and \code{(s/min \opt{\var{lt}})}.

\defineentry{s/map-uniq}
\begin{procedure}
  \code{(s/map-uniq \var{procedure} \opt{\var{hash} \var{equiv?}})}
\end{procedure}
\returns{} a transformer

The \code{s/map-uniq} procedure returns the transformer composition of \code{(s/map
  \var{procedure})} and \code{(s/uniq \opt{\var{hash} \var{equiv?}})}.

\defineentry{s/max}
\begin{procedure}
  \code{(s/max \opt{\var{gt}})}
\end{procedure}
\returns{} a transformer

The \code{s/max} procedure returns a transformer that returns the maximum value in a
finite stream according to greater-than operator \var{gt}. If \var{gt} is omitted, numeric
greater-than (\code{>}) is used. If the stream contains multiple equivalent maxima, the
first is returned. If the stream is empty, it returns false.

\defineentry{s/max-by}
\begin{procedure}
  \code{(s/max-by \var{procedure} \opt{\var{gt}})}
\end{procedure}
\returns{} a transformer

The \code{s/max-by} procedure returns a transformer that returns the value \var{x} in a
finite stream that maximizes \code{(\var{procedure} \var{x})} according to greater-than
operator \var{gt}. If \var{gt} is omitted, numeric greater-than (\code{>}) is used. If the
stream contains multiple equivalent maxima, the first is returned. If the stream is empty,
it returns false.

\defineentry{s/mean}
\begin{procedure}
  \code{(s/mean \opt{\var{procedure}})}
\end{procedure}
\returns{} a transformer

The \code{s/mean} procedure returns a transformer that maps values of a finite stream with
\var{procedure} and returns the mean, or NaN if the stream is empty. If \var{procedure} is
omitted, the identity procedure is used.

\defineentry{s/min}
\begin{procedure}
  \code{(s/min \opt{\var{lt}})}
\end{procedure}
\returns{} a transformer

The \code{s/min} procedure returns a transformer that returns the minimum value in a
finite stream according to less-than operator \var{lt}. If \var{lt} is omitted, numeric
less-than (\code{<}) is used. If the stream contains multiple equivalent minima, the first
is returned. If the stream is empty, it returns false.

\defineentry{s/min-by}
\begin{procedure}
  \code{(s/min-by \var{procedure} \opt{\var{lt}})}
\end{procedure}
\returns{} a transformer

The \code{s/min-by} procedure returns a transformer that returns the value \var{x} in a
finite stream that minimizes \code{(\var{procedure} \var{x})} according to less-than
operator \var{lt}. If \var{lt} is omitted, numeric less-than (\code{<}) is used. If the
stream contains multiple equivalent minima, the first is returned. If the stream is empty,
it returns false.

\defineentry{s/none?}
\begin{procedure}
  \code{(s/none? \opt{\var{procedure}})}
\end{procedure}
\returns{} a transformer

The \code{s/none?} procedure returns a transformer that returns true if and only if
the stream is empty. If \var{procedure} is provided, the stream is first filtered
with \var{procedure}.

\defineentry{s/reverse}
\begin{variable}
  \code{s/reverse}
\end{variable}
\antipar

The \code{s/reverse} variable is a transformer that reverses a finite stream and returns
its values in a list.

\defineentry{s/sort}
\begin{procedure}
  \code{(s/sort \var{lt})}
\end{procedure}
\returns{} a transformer


The \code{s/sort} procedure returns a transformer that sorts a finite stream by less-than
operator \var{lt}. It returns a list of the sorted values.

\codebegin
> (s/> '(1 3 2) (s/sort <))
(1 2 3)
> (s/> '(1 3 2) (s/sort >))
(3 2 1)
\codeend

\defineentry{s/sort-by}
\begin{procedure}
  \code{(s/sort-by \var{procedure} \var{lt})}
\end{procedure}
\returns{} a transformer

The \code{s/sort-by} procedure returns a transformer that sorts a finite stream by
less-than operator \var{lt} according to \code{(\var{procedure} \var{x})} for each stream
value \var{x}. It returns a list of the sorted values.

\codebegin
> (s/> '(-3 -2 1 4) (s/sort-by abs <))
(1 -2 -3 4)
\codeend

\defineentry{s/stream}
\begin{variable}
  \code{s/stream}
\end{variable}
\antipar

The \code{s/stream} variable is a transformer that boxes its input. Using \code{s/stream}
as the last transformer in a pipeline effectively prevents \code{s/>} from attempting to
convert the result from stream to list, which can improve the performance of nested pipelines.

\codebegin
> (s/> chapters
    (s/map-concat
     (lambda (c)
       (s/> (chapter-sections c)
         (s/map (lambda (s) (list (chapter-name c) s)))
         s/stream))))
(("Erlang Embedding" "Introduction")
 ("Erlang Embedding" "Tuples")
 ("Stream Library" "Introduction")
 ("Stream Library" "Transformers"))
\codeend

\defineentry{s/sum}
\begin{variable}
  \code{s/sum}
\end{variable}
\antipar

The \code{s/sum} variable is a transformer that returns the sum of values in a finite
stream.

\defineentry{s/take}
\begin{procedure}
  \code{(s/take $n$)}
\end{procedure}
\returns{} a transformer

The \code{s/take} procedure returns a transformer that limits a stream to the first $n$
values. $n$ must be an exact integer. If $n$ is negative, it returns the empty stream.

\defineentry{s/take-last}
\begin{procedure}
  \code{(s/take-last $n$)}
\end{procedure}
\returns{} a transformer

The \code{s/take-last} procedure returns a transformer that returns a stream containing
the last $n$ values in a finite input stream. $n$ must be an exact integer. If $n$ is
negative, it returns the empty stream. If $n$ is greater than or equal to the length of
the stream, it returns the input stream unchanged.

\defineentry{s/take-while}
\begin{procedure}
  \code{(s/take-while \var{procedure})}
\end{procedure}
\returns{} a transformer

The \code{s/take-while} procedure returns a transformer that returns the stream of values
up to but not including the first one for which \var{procedure} returns false. If
\var{procedure} returns a true value for all values, it returns the input stream
unchanged.

\defineentry{s/uniq}
\begin{procedure}
  \code{(s/uniq \opt{\var{hash} \var{equiv?}})}
\end{procedure}
\returns{} a transformer

The \code{s/uniq} procedure returns a transformer that filters out all but the first
instance of each duplicate value, preserving order. Duplicates are determined by hash
function \var{hash} and equivalence function \var{equiv?}, as would be passed to
\code{make-hashtable}. If \var{hash} and \var{equiv?} are not provided, a hashtable
appropriate for the first key is created (see Table~\ref{tab:stream-ht}).

\defineentry{s/uniq-by}
\begin{procedure}
  \code{(s/uniq-by \var{procedure} \opt{\var{hash} \var{equiv?}})}
\end{procedure}
\returns{} a transformer

The \code{s/uniq-by} procedure returns a transformer that filters out all but the first
instance of each value \var{x} that duplicates \code{(\var{procedure} \var{x})},
preserving order. Duplicates are determined by hash function \var{hash} and equivalence
function \var{equiv?}, as would be passed to \code{make-hashtable}. If \var{hash} and
\var{equiv?} are not provided, a hashtable appropriate for the first key is created (see
Table~\ref{tab:stream-ht}).

\codebegin
> (s/> '(-1 2 1) (s/uniq-by abs))
(-1 2)
\codeend

\section {Transformer Helpers}

The following procedures and forms may be helpful in creating new transformers.

\defineentry{eos}
\begin{variable}
  \code{eos}
\end{variable}
\antipar

The \code{eos} variable is the end-of-stream object.

\defineentry{eos?}
\begin{procedure}
  \code{(eos? \var{x})}
\end{procedure}
\returns{} a boolean

The \code{eos?} procedure returns true if and only if \var{x} is the end-of-stream object.

\defineentry{define-stream-transformer}
\begin{syntax}
  \code{(define-stream-transformer (\var{name} \var{s} \var{a} \ldots) $e_1$ $e_2$ \ldots)}
\end{syntax}
\expandsto{} a definition

The \code{define-stream-transformer} form defines a transformer or transformer constructor
with the given \var{name} and arguments \var{a} \ldots. For each transformer input
\var{s}, \var{s} is rebound to \code{(require-stream \var{s})} and \var{a} \ldots\ are
rebound to \var{a} \ldots. This makes it safe for expressions $e_1$ $e_2$ \ldots\ to
assume \var{s} is a stream and to \code{set!} \var{a} \ldots\ while preserving transformer
semantics. Expressions $e_1$ $e_2$ \ldots\ should evaluate to the transformed stream.

If there are one or more arguments \var{a} \ldots, \code{define-stream-transformer}
defines a transformer constructor.

\codebegin
> (define-stream-transformer (s/take s n)
    (lambda ()
      (if (<= n 0)
          eos
          (let ([x (s)])
            (if (eos? x)
                eos
                (begin
                  (set! n (- n 1))
                  x))))))
> (s/> (numbers-from 1) (s/take 3))
(1 2 3)
\codeend

If there are zero arguments \var{a} \ldots, \code{define-stream-transformer} defines a
transformer variable.

\codebegin
> (define-stream-transformer (s/first s)
    (let ([x (s)])
      (and (not (eos? x)) x)))
> (s/> (numbers-from 1) s/first)
1
\codeend

Avoid creating a transformer that returns a zero-argument procedure that is not a stream,
because this library cannot distinguish it from a stream.

\defineentry{empty-stream}
\begin{variable}
  \code{empty-stream}
\end{variable}
\antipar

The \code{empty-stream} variable is the empty stream.

\defineentry{stream-cons}
\begin{procedure}
  \code{(stream-cons \var{x} \var{stream})}
\end{procedure}
\returns{} a stream

The \code{stream-cons} procedure returns a stream containing \var{x} followed by the
values of \var{stream}.

\defineentry{stream-cons*}
\begin{procedure}
  \code{(stream-cons* \var{x} \ldots\ \var{stream})}
\end{procedure}
\returns{} a stream

The \code{stream-cons*} procedure returns a stream containing \var{x} \ldots\ followed by
the values of \var{stream}.

\defineentry{require-stream}
\begin{procedure}
  \code{(require-stream \var{x})}
\end{procedure}
\returns{} a stream

The \code{require-stream} procedure returns \var{x} if it is a stream. If \var{x} is a
list, \code{require-stream} returns \code{(list->stream \var{x})}. If \var{x} is a vector,
it returns \code{(vector->stream \var{x})}. If \var{x} is a hashtable, it returns
\code{(hashtable->stream \var{x})}. Otherwise, error \code{\#(invalid-stream \var{x})} is
raised.

\defineentry{transformer-compose}
\begin{procedure}
  \code{(transformer-compose \var{a} \var{b})}
\end{procedure}
\returns{} a transformer

The \code{transformer-compose} procedure returns a new transformer that applies
transformer \var{a} followed by transformer \var{b}.

\defineentry{transformer-compose*}
\begin{procedure}
  \code{(transformer-compose* \var{ts})}
\end{procedure}
\returns{} a transformer

The \code{transformer-compose*} procedure returns a new transformer that applies the
transformers in list \var{ts} in order.
