% Copyright 2018 Beckman Coulter, Inc.
%
% Permission is hereby granted, free of charge, to any person
% obtaining a copy of this software and associated documentation files
% (the "Software"), to deal in the Software without restriction,
% including without limitation the rights to use, copy, modify, merge,
% publish, distribute, sublicense, and/or sell copies of the Software,
% and to permit persons to whom the Software is furnished to do so,
% subject to the following conditions:
%
% The above copyright notice and this permission notice shall be
% included in all copies or substantial portions of the Software.
%
% THE SOFTWARE IS PROVIDED "AS IS", WITHOUT WARRANTY OF ANY KIND,
% EXPRESS OR IMPLIED, INCLUDING BUT NOT LIMITED TO THE WARRANTIES OF
% MERCHANTABILITY, FITNESS FOR A PARTICULAR PURPOSE AND
% NONINFRINGEMENT. IN NO EVENT SHALL THE AUTHORS OR COPYRIGHT HOLDERS
% BE LIABLE FOR ANY CLAIM, DAMAGES OR OTHER LIABILITY, WHETHER IN AN
% ACTION OF CONTRACT, TORT OR OTHERWISE, ARISING FROM, OUT OF OR IN
% CONNECTION WITH THE SOFTWARE OR THE USE OR OTHER DEALINGS IN THE
% SOFTWARE.

\lstdefinelanguage{json}{
  basicstyle=\upshape\mdseries\frenchspacing\ttfamily,
  columns=flexible,
  string=[s]{"}{"},
  stringstyle=\color{blue},
  comment=[l]{:},
  commentstyle=\color{black},
  escapeinside={|}{|},
  frame=single
}

\chapter {JSON Interface}

\section {Introduction}

The programming interface includes procedures for JavaScript Object Notation
(JSON)~\cite{RFC7159}. In order to support all floating-point values, it extends the
specification with Infinity, -Infinity, and NaN.

This implementation translates JavaScript types into the following
Scheme types:

\begin{tabular}{ll}
  JavaScript & Scheme \\ \hline

  \code{true} & \code{\#t} \\
  \code{false} & \code{\#f} \\
  \code{null} & symbol \code{null} \\
  \code{Infinity} & \code{+inf.0} \\
  \code{-Infinity} & \code{-inf.0} \\
  \code{NaN} & \code{+nan.0} \\
  \var{string} & \var{string} \\
  \var{number} & \var{number} \\
  \var{array} & \var{list} \\
  \var{object} & hashtable mapping symbols to values \\

  \hline
\end{tabular}

This implementation does not range check values to ensure that a
JavaScript implementation can interpret the data.

\section {Programming Interface}

\defineentry{json:extend-object}
\begin{syntax}
  \code{(json:extend-object \var{ht} [\var{key} \var{value}] \etc)}
\end{syntax}

The \code{json:extend-object} construct adds the \var{key} /
\var{value} pairs to the hashtable \var{ht} using
\code{hashtable-set!}. Each \var{key} is a literal identifier
or an unquoted expression \code{,\var{e}} that evaluates to a
symbol. The resulting expression returns \var{ht}.

\defineentry{json:make-object}
\begin{syntax}
  \code{(json:make-object [\var{key} \var{value}] \etc)}
\end{syntax}

The \code{json:make-object} construct expands into a call to
\code{json:extend-object} with a new hashtable.

\defineentry{json:object?}
\begin{procedure}
  \code{(json:object? \var{x})}
\end{procedure}
\returns{} a boolean

The \code{json:object?} procedure determines whether or not the datum
\var{x} is an object created by \code{json:make-object}.

\defineentry{json:cells}
\begin{procedure}
  \code{(json:cells \var{ht})}
\end{procedure}
\returns{} a vector

The \code{json:cells} procedure returns a vector containing the
cells of the underlying hashtable.

\defineentry{json:size}
\begin{procedure}
  \code{(json:size \var{ht})}
\end{procedure}
\returns{} an integer

The \code{json:size} procedure returns the number of cells
in the underlying hashtable.

\defineentry{json:delete"!}
\begin{procedure}
  \code{(json:delete! \var{ht} \var{path})}
\end{procedure}
\returns{} unspecified

The \code{json:delete!} procedure expects \var{path} to be a symbol or
a non-empty list of symbols.
If \var{path} is a symbol, then \code{json:delete!} is equivalent
to \code{hashtable-delete!}.
Otherwise, \code{json:delete!} follows \var{path} as it descends into
the nested hashtable \var{ht}, treating each element as a key into
the hashtable reached by traversing the preceding elements.
When \code{json:delete!} reaches the final key in \var{path},
it calls \code{hashtable-delete!} to remove the association for
that key in the hashtable reached at that point.
If any key along the way does not map to a hashtable,
\code{json:delete!} has no effect.

\defineentry{json:ref}
\begin{procedure}
  \code{(json:ref \var{ht} \var{path} \var{default})}
\end{procedure}
\returns{} the value found by traversing \var{path} in \var{ht},
\var{default} if none

The \code{json:ref} procedure expects \var{path} to be a symbol or
a non-empty list of symbols.
If \var{path} is a symbol, then \code{json:ref} is equivalent
to \code{hashtable-ref}.
Otherwise, \code{json:ref} follows \var{path} as it descends into
the nested hashtable \var{ht}, treating each element as a key into
the hashtable reached by traversing the preceding elements.
When \code{json:ref} reaches the final key in \var{path},
it calls \code{hashtable-ref} to retrieve the value of that key
in the hashtable reached at that point.
If any key along the way does not map to a hashtable,
or if the final hashtable does not contain the final key,
\code{json:ref} returns \var{default}.

\defineentry{json:set"!}
\begin{procedure}
  \code{(json:set! \var{ht} \var{path} \var{value})}
\end{procedure}
\returns{} unspecified

The \code{json:set!} procedure expects \var{path} to be a symbol or
a non-empty list of symbols.
If \var{path} is a symbol, then \code{json:set!} is equivalent
to \code{hashtable-set!}.
Otherwise, \code{json:set!} follows \var{path} as it descends into
the nested hashtable \var{ht}, treating each element as a key into
the hashtable reached by traversing the preceding elements.
When \code{json:set!} reaches the final key in \var{path},
it calls \code{hashtable-set!} to set that key in the
hashtable reached at that point.
If any key along the way does not map to a hashtable,
\code{json:set!} installs an empty hashtable at that key
before proceding.
If \var{path} is malformed at some point, \code{json:set!} may
still mutate hashtables along the valid portion of the path
before reporting an error.

\defineentry{json:update"!}
\begin{procedure}
  \code{(json:update! \var{ht} \var{path} \var{procedure} \var{default})}
\end{procedure}
\returns{} unspecified

The \code{json:update!} procedure expects \var{path} to be a symbol or
a non-empty list of symbols.
If \var{path} is a symbol, then \code{json:update!} is equivalent
to \code{hashtable-update!}.
Otherwise, \code{json:update!} follows \var{path} as it descends into
the nested hashtable \var{ht}, treating each element as a key into
the hashtable reached by traversing the preceding elements.
When \code{json:update!} reaches the final key in \var{path},
it calls \code{hashtable-update!} to update that key in the
hashtable reached at that point.
If any key along the way does not map to a hashtable,
\code{json:update!} installs an empty hashtable at that key
before proceding.
If \var{path} is malformed at some point, \code{json:update!} may
still mutate hashtables along the valid portion of the path
before reporting an error.

\defineentry{json:read}
\begin{procedure}
  \code{(json:read \var{ip} \opt{\var{custom-inflate}})}
\end{procedure}
\returns{} a Scheme object or the eof object

The \code{json:read} procedure reads characters from the textual
input port \var{ip} and returns an appropriate Scheme object.
When \code{json:read} encounters a JSON object, it builds
the corresponding hashtable and calls \var{custom-inflate}
to perform application-specific conversion.
By default, \var{custom-inflate} is the identity function.

The following exceptions may be raised:
\begin{itemize}
\item \code{invalid-surrogate-pair}
\item \code{unexpected-eof}
\item \code{\#(unexpected-input \var{data} \var{input-position})}
\end{itemize}

\defineentry{json:write}
\begin{procedure}
  \code{(json:write \var{op} \var{x} \opt{\var{indent}} \opt{\var{custom-write}})}
\end{procedure}
\returns{} unspecified

The \code{json:write} procedure writes the object \var{x} to the
textual output port \var{op} in JSON format. JSON objects are sorted
by key using \code{string<?} on the string values of the symbols to
provide stable output. Scheme fixnums, bignums, and finite flonums may
be used as numbers.

When \var{indent} is a non-negative fixnum, the output is more
readable by a human. List items and key/value pairs are indented on
individual lines by the specified number of spaces. When \var{indent}
is 0, a newline is added to the end of the output. The default indent
of \code{\#f} produces compact output.

The optional \var{custom-write} procedure may intervene to handle
lists and hashtables differently or to handle objects that have no
direct JSON counterpart.  If \var{custom-write} does not handle a
given object, it should return false to let \code{json:write} proceed
normally.  The \var{custom-write} procedure is called with four
arguments: the textual output port \var{op}, the Scheme object
\var{x}, the current \var{indent} level, and a writer procedure
\var{wr} that should be used to write the values of arbitrary Scheme
objects.  The \var{wr} procedure is equivalent to
\code{(lambda (\var{op} \var{x} \var{indent}) (json:write \var{op} \var{x} \var{indent} \var{custom-write}))}.

If an object cannot be formatted, \code{\#(invalid-datum \var{x})}
is raised.

\defineentry{json:write-object}
\begin{syntax}
  \code{(json:write-object \var{op} \var{indent} \var{wr} [\var{key} \var{value}] \etc)}
\end{syntax}
\returns{} \code{\#t}

Given a textual output port \var{op}, an \var{indent} level, and a
writer procedure \var{wr}, the \code{json:write-object} construct
writes a JSON object with the given \var{key} / \var{value} pairs to
\var{op}, sorted by key using \code{string<?} on the string values of
the symbols.  Each \var{key} must be a distinct symbol.  The \var{wr}
procedure takes \var{op}, an object \var{x}, and an \var{indent} level
just like the \var{wr} procedure that is passed to \code{json:write}'s
\var{custom-write} procedure.

The following are equivalent, provided the keys are symbols.

\antipar\begin{alltt}
(begin (json:write \var{op} (json:make-object [\var{key} \var{value}] \etc) \var{indent}) \#t)
(json:write-object \var{op} \var{indent} json:write [\var{key} \var{value}] \etc)\end{alltt}\antipar

The latter trades code size and compile time for run-time efficiency.
At compile time, \code{json:write-object} sorts the keys and
preformats the strings that will separate values.

\defineentry{json:object->bytevector}
\begin{procedure}
  \code{(json:object->bytevector \var{x} \opt{\var{indent}} \opt{\var{custom-write}})}
\end{procedure}
\returns{} a bytevector

The \code{json:object->bytevector} procedure calls \code{json:write}
on \var{x} with the optional \var{indent} and \var{custom-write}, if
any, using a bytevector output port transcoded using
\code{(make-utf8-transcoder)} and returns the resulting bytevector.

\defineentry{json:bytevector->object}
\begin{procedure}
  \code{(json:bytevector->object \var{x} \opt{\var{custom-inflate}})}
\end{procedure}
\returns{} a Scheme object

The \code{json:bytevector->object} procedure creates a bytevector input port
on \var{x}, calls \code{json:read} with the optional \var{custom-inflate},
if any, and returns the resulting Scheme object after making sure the rest
of the bytevector is only whitespace.

\defineentry{json:object->string}
\begin{procedure}
  \code{(json:object->string \var{x} \opt{\var{indent}} \opt{\var{custom-write}})}
\end{procedure}
\returns{} a JSON formatted string

The \code{json:object->string} procedure creates a string output port,
calls \code{json:write} on \var{x} with the optional \var{indent} and
\var{custom-write}, if any, and returns the resulting string.

\defineentry{json:string->object}
\begin{procedure}
  \code{(json:string->object \var{x} \opt{\var{custom-inflate}})}
\end{procedure}
\returns{} a Scheme object

The \code{json:string->object} procedure creates a string input port
on \var{x}, calls \code{json:read} with the optional \var{custom-inflate},
if any, and returns the resulting Scheme object after making sure the rest
of the string is only whitespace.

\defineentry{json:write-structural-char}
\begin{procedure}
  \code{(json:write-structural-char \var{x} \var{indent} \var{op})}
\end{procedure}
\returns{} the new indent level

The \code{json:write-structural-char} procedure writes the character
\var{x} at an appropriate \var{indent} level to the textual output
port \var{op}. The character should be one of the following JSON
structural characters: \code{[ ] \{ \} : ,}

This procedure is intended for use within custom writers passed in to
\code{json:write} and, for performance, it does not check its input arguments.

\defineentry{stack->json}
\begin{procedure}
  \code{(stack->json \var{k} \opt{\var{max-depth}})}
\end{procedure}
\returns{} a JSON object

The \code{stack->json} procedure renders the stack of continuation \var{k}
as a JSON object by calling \hyperlink{walk-stack}{\code{walk-stack}}.
The return value may contain the following keys:

\begin{tabular}{lp{4.6in}}
  \code{type} & \code{"stack"} \\
  \code{depth} & the depth of the stack \\
  \code{truncated} & if present, the \var{max-depth} at which the stack dump was truncated \\
  \code{frames} & if present, a list of JSON objects representing stack frames
\end{tabular}

A stack frame may contain the following keys:

\begin{tabular}{lp{4.6in}}
  \code{type} & \code{"stack-frame"} \\
  \code{depth} & the depth of this frame \\
  \code{source} & if present, a source object for the return point \\
  \code{procedure-source} & if present, a source object for the procedure containing the return point \\
  \code{free} & if present, a list of JSON objects representing free variables
\end{tabular}

A source object \var{x} with source file descriptor \var{sfd} is
represented by a JSON object containing the following keys:

\begin{tabular}{lp{4.6in}}
  \code{bfp} & \code{(source-object-bfp \var{x})} \\
  \code{efp} & \code{(source-object-efp \var{x})} \\
  \code{path} & \code{(source-file-descriptor-path \var{sfd})} \\
  \code{checksum} & \code{(source-file-descriptor-checksum \var{sfd})}
\end{tabular}

A free variable with value \var{val} is represented by a JSON object
containing the following keys:

\begin{tabular}{lp{4.6in}}
  \code{name} & a string containing the variable name or its index \\
  \code{value} & the result of \code{\fixtilde(format "~s" \var{val})} \\
\end{tabular}

\defineentry{json-stack->string}
\begin{procedure}
  \code{(json-stack->string \opt{\var{op}} \var{x})}
\end{procedure}
\returns{} see below

The two argument form of \code{json-stack->string} prints the
stack represented by JSON object \var{x} to the textual output port \var{op}.
The single argument form of \code{json-stack->string} prints the stack
represented by JSON object \var{x} to a string output port and returns
the resulting string.
In either case, the printed form resembles that generated by \code{dump-stack}
except that source locations are given as file offsets rather than line and character
numbers.
