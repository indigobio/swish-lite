% Copyright 2025 Indigo BioAutomation, Inc.
%
% Permission is hereby granted, free of charge, to any person
% obtaining a copy of this software and associated documentation files
% (the "Software"), to deal in the Software without restriction,
% including without limitation the rights to use, copy, modify, merge,
% publish, distribute, sublicense, and/or sell copies of the Software,
% and to permit persons to whom the Software is furnished to do so,
% subject to the following conditions:
%
% The above copyright notice and this permission notice shall be
% included in all copies or substantial portions of the Software.
%
% THE SOFTWARE IS PROVIDED "AS IS", WITHOUT WARRANTY OF ANY KIND,
% EXPRESS OR IMPLIED, INCLUDING BUT NOT LIMITED TO THE WARRANTIES OF
% MERCHANTABILITY, FITNESS FOR A PARTICULAR PURPOSE AND
% NONINFRINGEMENT. IN NO EVENT SHALL THE AUTHORS OR COPYRIGHT HOLDERS
% BE LIABLE FOR ANY CLAIM, DAMAGES OR OTHER LIABILITY, WHETHER IN AN
% ACTION OF CONTRACT, TORT OR OTHERWISE, ARISING FROM, OUT OF OR IN
% CONNECTION WITH THE SOFTWARE OR THE USE OR OTHER DEALINGS IN THE
% SOFTWARE.

\newcommand{\nl}[1]{\textsf{\textit{#1}}}

\chapter {OOP Library}\label{chap:oop}

The \code{(swish oop)} library provides a simple object system with classes that support
single inheritance. It extends Scheme's record system with instance and virtual instance
methods. The virtual function table is stored in the record-type descriptor to avoid any
extra storage in each class instance.

% ----------------------------------------------------------------------------
\defineentry{define-class}

\begin{syntax}\begin{alltt}
(define-class \var{name}
  \opt{(parent \var{parent})}
  \opt{(fields (\set{immutable\alt{}mutable} \var{field}) \etc)}
  \opt{(protocol \var{protocol})}
  (method (\var{method} \var{formal} \etc) \var{body} \etc) \etc
  (virtual (\var{virtual} \var{formal} \etc) \var{body} \etc) \etc
  )\strut\end{alltt}
\end{syntax}
\expandsto{}\begin{alltt}\antipar
(module ((\var{name} \nl{implicit-export} \etc))
  (meta define ctcls \nl{make-ctcls})
  (define-syntax \var{name} (\nl{make-class-dispatcher} ctcls))
  (define-property \var{name} \nl{class-key} ctcls)
  (define ($\var{name}.\nl{virtual}.\nl{arity}.impl inst \var{formal} \etc)
    (\nl{open-instance} \var{name} inst \var{body} \etc)) \etc
  (define ($\var{name}.\nl{override}.\nl{arity}.impl inst \var{formal} \etc)
    (\nl{open-instance} \var{name} inst \var{body} \etc)) \etc
  (define \var{name}.rtd (\nl{make-class-rtd} \var{name}))
  (define \var{name}.rcd
    (make-record-constructor-descriptor \var{name}.rtd \nl{parent-rcd} \var{protocol}))
  (define \var{name}.make (record-constructor \var{name}.rcd))
  (define (\var{name}.\var{field} inst)
    (\nl{record-check} '\var{field} inst \var{name}.rtd)
    (\nl{object-ref} inst \nl{offset})) \etc
  (define (\var{name}.\nl{mutable-field}.set! inst val)
     (\nl{record-check} '\nl{mutable-field} inst \var{name}.rtd)
     (\nl{record-set!} inst \nl{offset} val)) \etc
  (define ($\var{name}.\var{method}.\nl{arity} inst \var{formal} \etc)
    (\nl{open-instance} \var{name} inst \var{body} \etc)) \etc
  (define (\var{name}.\var{method}.\nl{arity} inst \var{formal} \etc)
    (\nl{record-check} '\var{method} inst \var{name}.rtd)
    ($\var{name}.\var{method}.\nl{arity} inst \var{formal} \etc)) \etc
  (define ($\var{name}.\nl{virtual}.\nl{arity} inst \var{formal} \etc)
    ((\nl{object-ref} (record-rtd inst) \nl{offset}) inst \var{formal} \etc)) \etc
  (define (\var{name}.\nl{virtual}.\nl{arity} inst \var{formal} \etc)
    (\nl{record-check} '\nl{virtual} inst \var{name}.rtd)
    ($\var{name}.\nl{new-virtual}.\nl{arity} inst \var{formal} \etc)) \etc
  )\end{alltt}

The \code{define-class} macro defines a class, \var{name}, a macro described in
Figure~\ref{oop:class-name}. The optional \code{parent} clause specifies the parent
class. The class inherits all fields and methods from its parent.

Figure~\ref{oop:records} diagrams the layout of class records.

\begin{figure}[p]
  \begin{center}
    \includegraphics{oop-layout.pdf}
  \end{center}
  The record-type descriptor for the class record is extended with a field for each
  virtual method.
  \caption{\label{oop:records}Class records}
\end{figure}

The optional \code{fields} clause specifies the set of instance fields, which defaults to
the empty set. Within a class, no two fields may have the same name, and the set of field
names must be disjoint from the set of method names. A field shadows any inherited field
or method of the same name.

The \code{protocol} clause specifies the instance constructor in the form used by Scheme's
\code{define-record-type}.

Each \code{method} clause specifies a non-virtual instance method, and each \code{virtual}
clause specifies a virtual instance method. The body expressions are evaluated in
\nl{open-instance} scope, described in Figure~\ref{oop:open-instance}. Within a class, no
two instance methods may have the same name and arity, and the set of instance method
names must be disjoint from the set of field names. An instance method shadows any
inherited field of the same name. An instance method shadows---or overrides in the case of
virtual methods---any inherited instance method of the same name and arity.

No field or method may be named any of the following: \code{base}, \code{isa?},
\code{make}, and \code{this}.

\begin{figure}[p]
The \code{define-class} macro defines a \var{name} macro that behaves according to the
following chart, where \nl{field} includes every field and unshadowed inherited field,
\nl{mutable-field} includes every mutable \nl{field}, and \nl{method} includes every
instance method and unshadowed inherited instance method:

\begin{center}
\begin{tabular}{|l|l|}
\multicolumn{1}{l}{\textbf{syntax}} & \multicolumn{1}{l}{\textbf{returns}}\\
\hline
\code{(\var{name} isa?\ \var{e})} &
 \code{(record?\ \var{e} \var{name}.rtd)}\\
\code{(\var{name} make \var{arg} \etc)} &
 \code{(\var{name}.make \var{arg} \etc)}\\
\code{(\var{name} \nl{field} \var{inst})} &
 \code{(\nl{class}.\nl{field} \var{inst})}\\
\code{(\var{name} \nl{mutable-field} \var{inst} \var{val})} &
 \code{(\nl{class}.\nl{mutable-field}.set! \var{inst} \var{val})}\\
\code{(\var{name} \nl{method} \var{inst} \var{arg} \etc)} &
 \code{(\nl{class}.\nl{method}.\nl{arity} \var{inst} \var{arg} \etc)}\\
\hline
\end{tabular}
\end{center}
\caption{\label{oop:class-name}Description of class \var{name} macro}
\end{figure}

\begin{figure}[p]
The \nl{open-instance} macro introduces a scope with the following definitions, where
\nl{field} includes every field and unshadowed inherited field, \nl{mutable-field}
includes every mutable \nl{field}, \nl{method} includes every instance method and
unshadowed inherited instance method, \nl{base-field} includes every parent-class field
and unshadowed parent-class-inherited field, \nl{base-mutable-field} includes every
mutable \nl{base-field}, \nl{base-nonvirt} includes every parent-class non-virtual
instance method and unshadowed parent-class-inherited non-virtual instance method, and
\nl{base-virtual} includes every parent-class virtual instance method and unshadowed
parent-class-inherited virtual instance method.

\begin{center}
\begin{tabular}{|l|l|}
\multicolumn{1}{l}{\textbf{syntax}} & \multicolumn{1}{l}{\textbf{returns}}\\
\hline
\code{this} &
  \code{inst}\\
\code{(this \nl{field})} &
  \code{(\nl{object-ref} inst \nl{offset})}\\
\code{(this \nl{mutable-field} \var{val})} &
  \code{(\nl{object-set!} inst \nl{offset} \var{val})}\\
\code{(this \nl{method} \var{arg} \etc)} &
  \code{(\$\nl{class}.\nl{method}.\nl{arity} inst \var{arg} \etc)}\\
\code{(base \nl{base-field})} &
  \code{(\nl{object-ref} inst \nl{offset})}\\
\code{(base \nl{base-mutable-field} \var{val})} &
  \code{(\nl{object-set!} inst \nl{offset} \var{val})}\\
\code{(base \nl{base-nonvirt} \var{arg} \etc)} &
  \code{(\$\nl{class}.\nl{base-nonvirt}.\nl{arity} inst \var{arg} \etc)}\\
\code{(base \nl{base-virtual} \var{arg} \etc)} &
  \code{(\$\nl{class}.\nl{base-virtual}.\nl{arity}.impl inst \var{arg} \etc)}\\
\nl{field} &
  \code{(\nl{object-ref} inst \nl{offset})}\\
\code{(set!\ \nl{mutable-field} \var{val})} &
  \code{(\nl{object-set!} inst \nl{offset} \var{val})}\\
\code{(\nl{method} \var{arg} \etc)} &
  \code{(\$\nl{class}.\nl{method}.\nl{arity} inst \var{arg} \etc)}\\
\hline
\end{tabular}
\end{center}
\caption{\label{oop:open-instance}Description of \nl{open-instance} scope}
\end{figure}
