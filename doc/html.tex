% Copyright 2018 Beckman Coulter, Inc.
%
% Permission is hereby granted, free of charge, to any person
% obtaining a copy of this software and associated documentation files
% (the "Software"), to deal in the Software without restriction,
% including without limitation the rights to use, copy, modify, merge,
% publish, distribute, sublicense, and/or sell copies of the Software,
% and to permit persons to whom the Software is furnished to do so,
% subject to the following conditions:
%
% The above copyright notice and this permission notice shall be
% included in all copies or substantial portions of the Software.
%
% THE SOFTWARE IS PROVIDED "AS IS", WITHOUT WARRANTY OF ANY KIND,
% EXPRESS OR IMPLIED, INCLUDING BUT NOT LIMITED TO THE WARRANTIES OF
% MERCHANTABILITY, FITNESS FOR A PARTICULAR PURPOSE AND
% NONINFRINGEMENT. IN NO EVENT SHALL THE AUTHORS OR COPYRIGHT HOLDERS
% BE LIABLE FOR ANY CLAIM, DAMAGES OR OTHER LIABILITY, WHETHER IN AN
% ACTION OF CONTRACT, TORT OR OTHERWISE, ARISING FROM, OUT OF OR IN
% CONNECTION WITH THE SOFTWARE OR THE USE OR OTHER DEALINGS IN THE
% SOFTWARE.

\lstdefinelanguage{json}{
  basicstyle=\upshape\mdseries\frenchspacing\ttfamily,
  columns=flexible,
  string=[s]{"}{"},
  stringstyle=\color{blue},
  comment=[l]{:},
  commentstyle=\color{black},
  escapeinside={|}{|},
  frame=single
}

\chapter {HTML Interface}\label{chap:http}

\section {Introduction}

The programming interface includes procedures for the HyperText Markup
Language (HTML) version 5~\cite{html5}.

\section {Programming Interface}

\defineentry{html:encode}
\begin{procedure}
  \code{(html:encode \var{s})} \\
  \code{(html:encode \var{op} \var{s})}\strut
\end{procedure}
\returns{} see below

The \code{html:encode} procedure converts special character entities
in string \var{s}.

\begin{tabular}{ll}
  input & output \\ \hline
  \code{"} & \code{\&quot;} \\
  \code{\&} & \code{\&amp;} \\
  \code{\textless} & \code{\&lt;} \\
  \code{\textgreater} & \code{\&gt;} \\
  \hline
\end{tabular}

The single argument form of \code{html:encode} returns an encoded
string.

The two argument form of \code{html:encode} sends the encoded string
to the textual output port \var{op}.

\defineentry{html->string}
\begin{procedure}
  \code{(html->string \var{x})} \\
  \code{(html->string \var{op} \var{x})}\strut
\end{procedure}
\returns{} see below

The \code{html->string} procedure transforms an object into
HTML. The transformation, $H$, is described below:

\begin{tabular}{ll}
  \var{x} & $H(\var{x})$\\ \hline

  \code{()} & nothing\\
  \code{\#!void} & nothing\\
  \code{\var{string}} & $E(\var{string})$\\
  \code{\var{number}} & \var{number}\\
  \code{(begin \var{pattern} \etc)} & $H(\var{pattern})$\etc\\
  \code{(cdata \var{string} \etc)} &
  \code{[!CDATA[\var{string}$\etc$]]}\\
  \code{(html5 \opt{(@ \var{attr} \etc)} \var{pattern} \etc)} &
  \code{<!DOCTYPE html><html $A(\var{attr})$ $\etc$>$H(\var{pattern})\etc$</html>}\\
  \code{(raw \var{string} \etc)} & \var{string}$\etc$\\
  \code{(script \opt{(@ \var{attr} \etc)} \var{string} \etc)} &
  \code{<script $A(\var{attr})$ $\etc$>\var{string}$\etc$</script>}\\
  \code{(style \opt{(@ \var{attr} \etc)} \var{string} \etc)} &
  \code{<style $A(\var{attr})$ $\etc$>\var{string}$\etc$</style>}\\
  \code{(\var{tag} \opt{(@ \var{attr} \etc)} \var{pattern} \etc)} &
  \code{<\var{tag} $A(\var{attr})$ $\etc$>$H(\var{pattern})\etc$</\var{tag}>}\\
  \code{(\var{void-tag} \opt{(@ \var{attr} \etc)})} &
  \code{<\var{void-tag} $A(\var{attr})$ $\etc$>}\\

  \hline
\end{tabular}

$E$ denotes the \code{html:encode} function.

For the \code{html5} tag, if there is no \var{attr} with \code{lang} as its key,
then $H$ acts as if the \var{attr} \code{(lang "en")} were specified.

A \var{void-tag} is one of \code{area}, \code{base}, \code{br},
\code{col}, \code{embed}, \code{hr}, \code{img},
\code{input}, \code{keygen}, \code{link}, \code{menuitem},
\code{meta}, \code{param}, \code{source}, \code{track}, or
\code{wbr}. A \var{tag} is any other symbol.

The attribute transformation, $A$, is described below, where \var{key}
is a symbol:

\begin{tabular}{ll}
  \var{attr} & $A(\var{attr})$\\ \hline

  \code{\#!void} & nothing\\
  \code{(\var{key})} & \var{key}\\
  \code{(\var{key} \var{string})} & \code{\var{key}="$E(\var{string})$"}\\
  \code{(\var{key} \var{number})} & \code{\var{key}="\var{number}"}\\

  \hline
\end{tabular}

The single argument form of \code{html->string} returns an encoded
HTML string.

The two argument form of \code{html->string} sends the encoded HTML
string to the textual output port \var{op}.

Input that does not match the specification causes a
\code{\#(bad-arg html->string \var{x})} exception to be raised.

\defineentry{html->bytevector}
\begin{procedure}
  \code{(html->bytevector \var{x})}
\end{procedure}
\returns{} a bytevector

The \code{html->bytevector} procedure calls \code{html->string} on
\var{x} using a bytevector output port transcoded using
\code{(make-utf8-transcoder)} and returns the resulting bytevector.
