% Copyright 2018 Beckman Coulter, Inc.
%
% Permission is hereby granted, free of charge, to any person
% obtaining a copy of this software and associated documentation files
% (the "Software"), to deal in the Software without restriction,
% including without limitation the rights to use, copy, modify, merge,
% publish, distribute, sublicense, and/or sell copies of the Software,
% and to permit persons to whom the Software is furnished to do so,
% subject to the following conditions:
%
% The above copyright notice and this permission notice shall be
% included in all copies or substantial portions of the Software.
%
% THE SOFTWARE IS PROVIDED "AS IS", WITHOUT WARRANTY OF ANY KIND,
% EXPRESS OR IMPLIED, INCLUDING BUT NOT LIMITED TO THE WARRANTIES OF
% MERCHANTABILITY, FITNESS FOR A PARTICULAR PURPOSE AND
% NONINFRINGEMENT. IN NO EVENT SHALL THE AUTHORS OR COPYRIGHT HOLDERS
% BE LIABLE FOR ANY CLAIM, DAMAGES OR OTHER LIABILITY, WHETHER IN AN
% ACTION OF CONTRACT, TORT OR OTHERWISE, ARISING FROM, OUT OF OR IN
% CONNECTION WITH THE SOFTWARE OR THE USE OR OTHER DEALINGS IN THE
% SOFTWARE.

\chapter {Testing}

The \code{(swish mat)} library provides methods to
define, iterate through, and run test cases, and to log the
results.

Test cases are called \emph{mats} and consist of a name, a
set of tags, and a test procedure of no arguments.
The set of mats is stored in
reverse order in a single, global list.
The list of tags allows the
user to group tests or mark them.  For example, tags can be used to
note that a test was created for a particular change request.

% ----------------------------------------------------------------------------
\defineentry{mat}\label{sec:mat}
\begin{syntax}
  \code{(mat \var{name} (\var{tag} \etc) $e_1$ $e_2$ \etc)}
\end{syntax}
\expandsto{} \code{(add-mat '\var{name} '(\var{tag} \etc)
  (lambda () $e_1$ $e_2$ \etc))}

The \code{mat} macro creates a mat with the given \var{name},
\var{tag}s, and test procedure $e_1$ $e_2$ \etc\ using the
\code{add-mat} procedure.

% ----------------------------------------------------------------------------
\defineentry{isolate-mat}
\begin{syntax}
\code{(isolate-mat \var{name} (\var{tag} \etc) $e_1$ $e_2$ \etc)}\\
\code{(isolate-mat \var{name} (settings [\var{key} \var{val}] \etc) $e_1$ $e_2$ \etc)}\strut
\end{syntax}
\expandsto{}
\code{(mat \var{name} (\var{tag} \etc) (\$isolate-mat (lambda () $e_1$ $e_2$ \etc)))}

Tests involving process operations, such as \code{spawn}, \code{send}, and
\code{receive}, should use \code{isolate-mat} in place of \code{mat}
to isolate the host system from the test code.
The \code{isolate-mat} macro is provided by the \code{(swish testing)} library,
which can be accessed via \code{swish-test}.

The \code{settings} form can be used within \code{isolate-mat} to override the
following defaults:

\begin{tabular}{lcp{.5\textwidth}}
  key & default & description \\ \hline
  \code{tags}    & \code{()} & a list of \var{tag} \etc \\
  \code{timeout} & 60000 & deadline for test to complete, in milliseconds \\
  \code{process-cleanup-deadline} & 500 & maximum time to wait for spawned processes to terminate, in milliseconds \\
  \code{process-kill-delay} & 100 & delay in milliseconds before killing each spawned process that did not terminate
  by the \code{process-cleanup-deadline}
\end{tabular}

% ----------------------------------------------------------------------------
\defineentry{mat:add-annotation"!}
\begin{syntax}
  \code{(mat:add-annotation! \var{expr})}
\end{syntax}
\returns{} unspecified

The \code{mat:add-annotation!} macro extends the list of annotations recorded in
the \code{meta-data} for the mat result of the current mat.
Each annotation records the \code{erlang:now} timestamp,
the source for the call site, if available, and
the value of \var{expr}, which must evaluate to a valid JSON datum.
This macro must be called only within the dynamic extent of a running mat.
See \hyperref[load-results]{\code{load-results}} for more information.

% ----------------------------------------------------------------------------
\defineentry{add-mat}
\begin{procedure}
  \code{(add-mat \var{name} \var{tags} \var{test})}
\end{procedure}
\returns{} unspecified

The \code{add-mat} procedure adds a mat to the front of the global
list. \var{name} is a symbol, \var{tags} is a list, and \var{test} is
a procedure of no arguments.

If \var{name} is already used, an exception is raised.

% ----------------------------------------------------------------------------
\defineentry{clear-mats}
\begin{procedure}
  \code{(clear-mats)}
\end{procedure}
\returns{} unspecified

The \code{clear-mats} procedure clears the global list of mats.

% ----------------------------------------------------------------------------
\defineentry{run-mat}
\begin{procedure}
  \code{(run-mat \var{name} \var{reporter})}
\end{procedure}
\returns{} see below

The \code{run-mat} procedure runs the mat of the given \var{name} by
executing its test procedure with an altered exception handler. If the
test procedure completes without raising an exception, the mat result
is \code{pass}. If the test procedure raises exception \var{e}, the
mat result is \code{(fail~.~\var{e})}.

After the mat completes, the \code{run-mat} procedure tail calls
\code{(\var{reporter} \var{name} \var{tags} \var{result} \var{statistics})}.

If no mat with the given \var{name} exists, an exception is raised.

% ----------------------------------------------------------------------------
\defineentry{run-mats}
\begin{syntax}
  \code{(run-mats \opt{\var{name}} \etc)}
\end{syntax}
\returns{} \code{\#t} if all mats pass, \code{\#f} otherwise

The \code{run-mats} macro runs each mat specified by symbols
\var{name} \etc.  When no names are supplied, all
mats are executed.  After each mat is executed, its result, name, and
exception if it failed are displayed.  When the mats are finished, a
summary of the number run, passed, and failed is displayed.

% ----------------------------------------------------------------------------
\defineentry{run-mats-from-file}
\begin{procedure}
  \code{(run-mats-from-file \var{filename})}
\end{procedure}
\returns{} \code{\#t} if all mats pass, \code{\#f} otherwise

The \code{run-mats-from-file} procedure clears the global list of mats, loads
\var{filename} into an isolated environment, and calls \code{(run-mats)}.

% ----------------------------------------------------------------------------
\defineentry{run-mats-to-file}
\begin{procedure}
  \code{(run-mats-to-file \var{filename})}
\end{procedure}
\returns{} unspecified

The \code{run-mats-to-file} procedure executes all mats and writes
the results into the file specified by the string \var{filename}. If
the file exists, its contents are overwritten. The file format is a
sequence of JSON objects readable with \code{load-results} and
\code{summarize}.

% ----------------------------------------------------------------------------
\defineentry{for-each-mat}
\begin{procedure}
  \code{(for-each-mat \var{procedure})}
\end{procedure}
\returns{} unspecified

The \code{for-each-mat} procedure calls \code{(\var{procedure}
  \var{name} \var{tags})} for each mat, in no particular order.

% ----------------------------------------------------------------------------
\defineentry{load-results}
\begin{procedure}
  \code{(load-results \var{filename})}
\end{procedure}
\returns{} a JSON object
\phantomsection\label{load-results}

The \code{load-results} procedure reads the contents of the file
specified by string \var{filename} and returns a JSON object
with the following keys:

\begin{tabular}{lp{3.6in}}
  \code{meta-data} & a JSON object \\
  \code{report-file} & \var{filename} \\
  \code{results} & a list of JSON objects \\
\end{tabular}

The \code{meta-data} object contains at least the following keys:

\begin{tabular}{lp{4.6in}}
  \code{completed} & \code{\#t} if test suite completed, \code{\#f} otherwise \\
  \code{hostname} & \code{(osi\_get\_hostname)} of the host system \\
  \code{machine-type} & \code{(machine-type)} of the host system \\
  \code{test-file} & name of the file containing the tests \\
  \code{test-run} & uuid generated by \code{swish-test} for the set of tests run \\
  \code{uname} & the result from \code{get-uname} as a JSON object containing
    \code{os-machine},
    \code{os-release},
    \code{os-system}, and
    \code{os-version} \\
\end{tabular}

If the test suite completed, the \code{meta-data} object also contains the
following keys:

\begin{tabular}{lp{4.6in}}
  \code{date} & \code{(format-rfc2822 (current-date))} at the start of the test suite \\
  \code{software-info} & \code{(software-info)} for the tested code \\
  \code{timestamp} & \code{(erlang:now)} at the start of the test suite \\
\end{tabular}

Each result is a JSON object with the following keys:

\newcolumntype{L}[1]{>{\raggedright\let\newline\\\arraybackslash\hspace{0pt}}p{#1}}
\begin{tabular}{lp{4.6in}}
  \code{message} & error message from failing test, or empty string \\
  \code{meta-data} & a JSON object that contains: \\
  & \begin{tabular}{lL{3.5in}}
      \code{annotations} & an ordered list of JSON objects recording data from
        \code{mat:add-annotation!}, \\
      \code{start-time} & \code{(erlang:now)} at the start of the test, and \\
      \code{end-time} & \code{(erlang:now)} at the end of the test \\
    \end{tabular} \\
  \code{sstats} & a JSON object representing the \code{sstats-difference} for the test \\
  \code{stacks} & if test failed with exception \var{e}, then\\
                & \code{(map stack->json (exit-reason->stacks \var{e}))} \\
  \code{tags} & a list of strings corresponding to the symbolic tags in the \code{mat} form \\
  \code{test} & a string corresponding to the symbolic \code{mat} name \\
  \code{test-file} & the name of the test file \\
  \code{type} & the type of result: \code{"pass"}, \code{"fail"}, \code{"skip"} \\
\end{tabular}

% ----------------------------------------------------------------------------
\defineentry{summarize}
\begin{procedure}
  \code{(summarize \var{files})}
\end{procedure}
\returns{} five values: the number of passing mats, the number of
failing mats, the number of skipped mats, the number of completed suites,
and the length of \var{files}.

The \code{summarize} procedure reads the contents of each file in
\var{files}, a list of string filenames, and returns the number of
passing mats, the number of failing mats, the number of skipped mats,
the number of completed test suites, and the number of files specified.
An error is raised if any
entry is malformed.
